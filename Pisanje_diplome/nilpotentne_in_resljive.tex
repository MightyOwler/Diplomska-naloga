\section{Nilpotentne in rešljive grupe}

\subsection{Definicija in osnovne lasntosti}

Intuitivno gledano so nilpotntne in rešljive grupe tiste, ki so po svoji strukturi še najbolj podobne Abelovim. Da jih lahko vpeljemo, moramo najprej uvesti nekaj pojmov.
\begin{definicija}\label{def_komutator_grup}
    Naj bo $G$ grupa in $H, K \le G$ njeni podgrupi. Potem definiramo \emph{komutator podgrup $H$ in $K$} kot podgrupo \begin{equation*}
        [H, K]  = \langle [h, k] | h \in H, k \in K \rangle.
    \end{equation*}
\end{definicija}

\begin{definicija}
\label{def_iztek_zaporedja}
Naj bo $G$ grupa in $(H_k)_{k \ge 1}$ padajoče zaporedje njenih podgrup, torej $H_{i + 1} \subseteq H_{i}$ za vsak $i \ge 1$. 
Rečemo, da se \emph{zaporedje $(H_k)_{k \ge 1}$ izteče z grupo $K$}, če obstaja naravno število $n$, da velja $H_k = K$ za vsako naravno število $k \ge n$.
\end{definicija}

\begin{definicija}
\label{def_nilpotentna_grupa}
Grupa $G$ je \emph{nilpotentna}, če se \emph{spodnja centralna vrsta} $(\gamma_k(G))_{k \ge 1}$, podana rekurzivno z \begin{equation*}
\gamma_1(G) := G \text{ in } \gamma_{k +1}(G) := [\gamma_k(G), G],
\end{equation*}  
izteče s trivialno grupo. Najmanjšemu številu $d$, za katero je $\gamma_{d + 1} = \{ 1_G \}$, rečemo \emph{razred nilpotentnosti grupe $G$}.    
\end{definicija}

Na kratko razmislimo, da je vsak člen spodnje centralne vrste $\gamma_k(G)$ edinka v grupi $G$. Osnovna ideja premisleka je dejstvo, da lahko konjugiranje prenesemo v notranjost komutatorja: Za poljubne elemente $g, h, k \in G$ velja zveza \begin{equation*}
    [g, h]^k = kghg^{-1}h^{-1}k^{-1} = kgk^{-1}khk^{-1}kg^{-1}k^{-1}kh^{-1}k^{-1} = [kgk^{-1}, khk^{-1}] = [g^k , h^k].
\end{equation*}
Elementi grupe $\gamma_1(G)  = [G , G]$ so po definiciji oblike $\prod_{i = 1}^n [g_i, h_i]$ za neke elemente $g_i, h_i \in G$, zato za vsak element $k \in G$ velja \begin{equation*}
    \left( \prod_{i = 1}^n [g_i, h_i]\right)^k = \prod_{i = 1}^n [g_i^k, h_i^k]. 
\end{equation*}
S tem smo dokazali, da je $\gamma_1(G)$ edinka v $G$, z indukcijo pokažemo enako za vse nadaljnje člene. S tem smo hkrati utemeljili tudi, da za vsako število $k \ge 0$ velja $\gamma_{k+1}(G) \triangleleft \gamma_k(G)$.

Celo družino primerov nilpotentnih grup nam podaja naslednja ugotovitev.
\begin{trditev}
\label{trd_p_grupe_so_nilpotentne}
    Vse $p$-grupe so nilpotentne. Natančneje, če je $\lvert G \rvert  = p^{d}$ za neko naravno število $d \ge 1$, potem je $G$ nilpotentna razreda največ $d$. 
\end{trditev}
\begin{dokaz}
    Naj bo $\lvert G \rvert = p^k$. Dokaz poteka z indukcijo po $k$. Za $k = 1$ je grupa Abelova in zato očitno nilpotentna. Za $k \ge 2$ uporabimo posledico razredne formule, da imajo $p$-grupe netrivialni center.
    Če je $Z(G) = G$ je grupa Abelova. Sicer sta po indukcijski predpostavki grupi $Z(G)$ in $G / Z(G)$ nilpotentni. Nilpotentnost kvocienta implicira obstoj najmanjšega števila $m$, za katero velja $\gamma_m(G) \subseteq Z(G)$.
    Od tod direktno sledi, da je $\gamma_{m + 1}(G) = \{ 1_G \}$.
\end{dokaz}

\begin{primer}
    Trditev \ref{trd_p_grupe_so_nilpotentne} nam sporoča, da so vse diedrske grupe oblike $D_{2 \cdot 2^{k}}$ nilpotentne. Izkaže se, da so to tudi edine. V nadaljevanju bomo namreč dokazali trditev \ref{trd_nilpotentne_so_produkti_sylowih}, ki pravi, da so nilpotentne grupe
    produkti svojih $p$-grup Sylowa. Ni težko premisliti, da je \begin{equation*}
        Z(D_{2m}) = \begin{cases}
            \{ 1_{D_{2m}} \}  ; \text{ če je $m$ liho}, \\
            \{ 1_{D_{2m}}, r^{m / 2} \} ;  \text{ če je $m$ sodo}.
        \end{cases}
    \end{equation*}
    Od tod sledi, da morajo biti nilpotentne diedrske grupe nujno oblike $D_{2 \cdot 2^{k}}$, saj bi jih sicer lahko razpisali kot produkt $p$-podgrup Sylowa, ki ima več elementov v jedru.
    \end{primer}

Pred dokazom napovedane trditve \ref{trd_nilpotentne_so_produkti_sylowih} navedimo naslednjo definicijo.

\begin{definicija}\label{def_normalizator}
    Naj bo $G$ grupa in $H$ njena podgrupa. Potem podgrupi \begin{equation*}
        N_G(H) := \{ g h g^{-1}  \vert  g \in G, h \in H \}
    \end{equation*}
    rečemo \emph{normalizator} podgrupe $H$ v grupi $G$. % V primeru, da je $N_G(H) = H$, je $H$ edinka v $G$ in pravimo, da $G$ \emph{normalizira} podgrupo $H$.
\end{definicija}

\begin{trditev}\label{trd_nilpotentne_so_produkti_sylowih}
    Končne nilpotentne grupe so direktni produkt svojih $p$-podgrup Sylowa.
\end{trditev}
\begin{dokaz}
    Dokaz je povzet po \cite{wikipedia2024nilpotent}.
    Naj bo $G$ nilpotentna grupa. Najprej pokažimo, da iz $H \lneq G$ sledi $H \propernormalsubgroup N_G(H)$. To storimo z indukcijo po moči grupe $G$.
    Če je $G$ Abelova, izjava očitno drži. Sicer je grupa $G$ nilpotentna razreda $1$ ali več, kar pomeni, da njen center ni trivialen; če je razred nilpotentnosti grupe $G$ na primer $d \ge 0$, velja \begin{equation*}
        \gamma_{d + 1} = [\gamma_d, G]  = 1 \Longrightarrow  \{ 1_G \} \neq \gamma_d \le Z(G) \Longrightarrow Z(G) \neq \{ 1_G \}.
    \end{equation*} 
     Zato obravnavamo dva primera. Če center $Z(G)$ ni vsebovan $H$, velja \\ $H \propernormalsubgroup N_G(H) Z(G)$ zaradi računa \begin{equation*}
        h^{h_Z g_Z} = h_Z g_Z h g_Z^{-1} h_Z^{-1} = h_Z h h_Z^{-1} \in H \, \, \, \text{ za vse } h, h_Z \in H, g_Z \in G.
    \end{equation*}
    Predpostavimo torej, da je $Z(G) \le H$. Potem po korespondenčnem izreku sledi, da je $H / Z(G)$ podgrupa nilpotentne grupe $G / Z(G)$.
    Ker center $Z(G)$ ni trivialen, uporabimo indukcijsko predpostavko na kvocientu $G / Z(G)$ in dobimo njeno podgrupo $K / Z(G)$, v kateri je $H/ Z(G)$ prava edinka.
    Ustrezno grupo za $H$ dobimo kot prasliko kanonične projekcije $\pi : G \to G / Z(G)$, uporabljeno na grupi $ K / Z(G)$.
    
    S pomočjo tega sklepa dokažimo, so $p$-podgrupe Sylowa v nilpotentni grupi $G$ ednike. Naj bodo $p_1, \ldots , p_s$ različna praštevila, ki delijo $\lvert G \rvert$ in naj bo $P_i$ poljubna $p_i$-podgrupa Sylowa grupe $G$ za vsak $1 \le i \le s$.
    Naj bo brez škode za splošnost $P := P_1$ in naj bo $N := N_G(P)$. Ker je $P$ edinka Sylowa v $N$, je v njej karakteristična. Od tod sledi, da je $P$ edinka tudi v $N_G(N)$, saj je zaradi $N \triangleleft N_G(N)$ konjugiranje z elementom iz $N_G(N)$
    avtomorfizem grupe $N$. Od tod sledi $N_G(N) \le N$ oziroma $N = N_G(N)$. Po ugotovitvi iz prejšnjega odstavka dobimo $N = G$, kar pomeni, da so $p$-podgrupe Sylowa nilpotentnih grup edinke.
    
    Od tod z indukcijo po moči grupe pokažemo, da lahko $G$ zapišemo kot direktni produkt svojih $p$-podgrup Sylowa. Pri tem se moramo zgolj
    sklicati na dejstvo, da za različni praštevili $p$ in $q$ velja $P_p \cap P_q = \{ 1_G \}$.
\end{dokaz}

\begin{posledica}\label{psl_ocena_razreda_nilpotentnosti}
    Nilpotentna grupa $G$ moči $n$ ali manj je razreda nilpotentnosti največ $\lfloor \log_2(n) \rfloor - 1$.
\end{posledica}
\begin{dokaz}
    Naj bo $G$ nilpotentna grupa moči $n$ ali manj. Po trditvi \ref{trd_nilpotentne_so_produkti_sylowih} je v skrajnem primeru enaka svoji 2-podgrupi Sylowa, katere moč je največ $2^{\log_2(n)}$ oziroma $2^{\lfloor \log_2(n) \rfloor}$, ker  mora biti moč celo število.
    To nam v luči dokaza trditve \ref{trd_p_grupe_so_nilpotentne} sporoča, da je razred nilpotentnosti grupe $G$ največ $\lfloor \log_2(G) \rfloor - 1$ (tu člen $-1$ prihaja iz definicije razreda nilpotentnosti, glej \ref{def_nilpotentna_grupa}).  
\end{dokaz}
\begin{definicija}

    \label{def_resljiva_grupa}
    Grupa $G$ je \emph{rešljiva}, če se \emph{izpeljana vrsta} $(G^{(k)})_{k \ge 0}$, podana rekurzivno z \begin{equation*}
        G^{(0)} := G \text{ in } G^{(k + 1)} := [G^{(k)}, G^{(k)}],
        \end{equation*}  
        izteče s trivialno grupo. Najmanjšemu številu $d$, za katero je $G^{(d)} = \{ 1_G \}$, rečemo \emph{razred rešljivosti grupe $G$}.    
    \end{definicija}

    Analogno kot pri nilpotentnih grupah sklepamo, da izpeljana vrsta $(G^{(k)})_{k \ge 0}$ tvori verigo edink, ki so vse hkrati tudi edinke v $G$.
    
    \begin{primer}
        Diedrske grupe $D_{2n} = \langle r , Z \rangle$ so rešljive razreda največ $2$. Z računom je enostavno pokazati, da je $(D_{2n})^{(1)}$ podmnožica Abelove podgrupe $\langle r \rangle$, zato bo grupa $(D_{2n})^{(2)} = ((D_{2n})^{(1)})^{(1)}$ trivialna.
        Ta sklep namiguje na nekatere lastnosti rešljivih grup, ki jih bomo obravnavali v trditvi \ref{trd_lastnosti_resljivih_grup}. 
    \end{primer}

    \begin{primer}
        Vse nilpotentne grupe so rešljive, saj za vsako število $k \ge 0$, saj velja $G^{(0)} = \gamma_0(G) = G$, z indukcijo sledi \begin{equation*}
        G^{(k)} = [G^{(k-1)}, G^{(k-1)}] \subseteq  [\gamma_{k -1}(G), G] = \gamma_{k}(G).
        \end{equation*}
        Niso pa vse nilpotentne grupe rešljive, primer so recimo diedrske grupe $D_{2n}$, kjer $2n$ ni dvojiška potenca. 
    \end{primer}
    
    \begin{trditev}
    \label{trd_lastnosti_resljivih_grup}
    Za rešljive grupe veljajo naslednje osnovne lasnosti.
     \begin{enumerate}
        \item Vsaka podgrupa rešljive grupe je rešljiva.
        \item Vsak kvocient rešljive grupe je rešljiv.
        \item Naj bo $N \triangleleft G$ in naj bosta $N$ in $G / N$ rešljivi grupi razreda $d_{N}$ oziroma $d_{G / N}$. Potem je $G$ rešljiva grupa razreda največ $d_N + d_{G / N}$.
        \item Naj bosta $M, N \triangleleft G$ rešljivi razreda $d_M$ oziroma $d_N$. Potem je edinka $MN$ rešljiva razreda največ $d_M + d_N$.   
     \end{enumerate}
    \end{trditev}
    \begin{dokaz}
        \begin{enumerate}
            \item To je očitna posledica dejstva, da za $H \le G$ velja $H^{(k)} \subseteq G^{(k)}$ za vsak $k \in \mathbb{N} \cup \left\{ 0\right\}$.
            \item Naj bo $G$ rešljiva in naj bo $N \triangleleft G$. Zaradi rešljivosti grupe $G$ obstaja naravno število $d$, da je $G^{k} \subseteq N$ za vse $k \ge d$, kar implicira $ (G / N)^{(k)} = \left\{ 1_{G / N}\right\}$ za vse $k \ge d$. 
            \item Ker je $G / N$ rešljiva grupa razreda $n_{G / N}$, bo $G^{(k)} \subseteq N$ za vse $k \ge d_{G / N}$. Ker je $N$ rešljiva razreda $d_N$, bo nadalje veljalo $G^{(k)} = \left\{ 1_G\right\}$ za vse $k \ge d_M + d_N$.
            \item Dokaz je prirejen po opombi 4 iz \cite[str.4]{Schneider_2016}. Po drugem izreku o izomorfizmu lahko zapišemo kratko eksaktno zaporedje \begin{equation*}
            \mathbf{1} \to M \to MN \to MN / M \cong N / (N \cap M) \to \mathbf{1}.
            \end{equation*}  
            Ker je $N$ rešljiva, je po drugi točki trditve  njen kvocient $ N / (N \cap M)$ rešljiv razreda največ $d_N$ in posledično tudi kvocient $ MN / M $. Ker je $M$ rešljiva razreda $d_M$, po tretji točki trditve sledi $(MN)^{(k)} = \left\{ 1_{G}\right\}$ za vse $k \ge d_N + d_M$.
        \end{enumerate}
    \end{dokaz}

    Razširitvena lema \ref{lem_razsiritvena_lema} nam ponuja naslednjo skromno oceno dolžine kratkih netrivialnih zakonov v rešljivih oziroma nilpotentnih grupah.

    \begin{trditev}
    \label{trd_ocitna_meja_za_kratke_zakone_resljive_grupe}
     Obstaja beseda $w \in F_2 = \langle a,  b \rangle$ dolžine $l(w) \le 4^{d}$, ki je zakon v vseh grupah razreda rešljivosti (ali nilpotentnosti) $d$ ali manj.  
    \end{trditev}
    \begin{dokaz}
        Trditev je posledica razširitvene leme \ref{lem_razsiritvena_lema}, dokaz poteka z indukcijo po razredu rešljivosti grupe $G$, ki ga označimo z $d$. Za $d = 1$ je grupa $G$ Abelova, zato je ustrezni zakon beseda $w = [a, b]$, ki je dolžine 4.
        Za $d > 1$ opazimo, da je kvocient $G / G^{(1)}$ Abelova grupa, $G^{(1)}$ pa rešljiva grupa razreda največ $d - 1$. Zato z uporabo razširitvene leme in indukcijske predpostavke najdemo besedo $w \in  F_2$ dolžine \begin{equation*}
            l(w) \le  4 \cdot 4^{d - 1} = 4^{d},
        \end{equation*}  
        ki je zakon v grupi $G$. Za nilpotentne grupe upoštevamo dejstvo $G^{(1)} \subseteq \gamma_1(G)$, kar implicira komutativnost grupe $G / \gamma_1(G)$ ($G^{(1)}$ je po definiciji najmanjša edinka, za katero je kvocient $G / G^{(1)}$ Abelova grupa). 
    \end{dokaz}
    
    \begin{opomba}
    V članku \cite[str.~8]{Kozma_Thom_2016} je podana nekoliko šibkejša meja $l(w) \le  4 \cdot 6^{d-1}$, ker je avtor uporabil šibkejšo obliko razširitvene leme.
    \end{opomba}

\subsection{Konstrukcija kratkih zakonov v nilpotentnih in rešljivih grupah}

Konstrukcija kratkih zakonov v nilpotentnih grupah je opisana v članku \cite{Elkasapy_Thom_2013} in z razlagami dopolnjena v magistrskem delu \cite{Schneider_2016}.
Glavna ideja je poiskati kratke netrivialne predstavnike izpeljane vrste proste grupe $F_2 = \langle a, b \rangle $. Najprej definirajmo zaporedji $(a_n)_n$ in $(b_n)_n$ v $F_2$ s predpisoma
\begin{equation*}
a_0 := a, \, a_{n + 1} := [b_n^{-1}, a_{n}] \text{ in } b_0 := b, \, b_{n + 1} := [a_{n}, b_{n}]. 
\end{equation*}  
Besede, ki jih bomo konstruirali s tema zaporedjema, morajo biti netrivialne. Zato potrebujemo naslednjo lemo (lema 3.1 v viru \cite{Kozma_Thom_2016} oziroma lema 8 v \cite{Schneider_2016}).
\begin{lema}
\label{lem_ni_krajsanj_produkti_ab}
Za vsako nenegativno celo število $n$ so besede $a_{n} a_{n}$, $a_{n}^{-1} a_{n}^{-1}$, $b_{n} b_{n}$, $b_{n}^{-1} b_{n}^{-1}$, $a_{n}^{-1} b_{n}$, $b_{n}^{-1} a_{n}$, $a_{n} b_{n}^{-1}$, $b_{n} a_{n}^{-1}$, $a_{n}^{-1} b_{n}^{-1}$ in $b_{n} a_{n}$ okrajšane.   
\end{lema}
\begin{dokaz}
    Dokaz poteka z indukcijo po $n$. Za $n = 0$ je trditev očitna, ker sta $a$ in $b$ različna generatorja grupe $F_2$. Za $n > 0$ najprej razpišimo produkt $a_n a_n$.
    \begin{equation*}
    a_{n} a_{n} = [b_{n- 1}^{-1}, a_{n-1}]^2 = b_{n- 1}^{-1} a_{n-1} b_{n- 1} \underbrace{a_{n-1}^{-1} b_{n- 1}^{-1}}_{\text{ni krajšanja}}  a_{n-1} b_{n- 1} a_{n-1}^{-1} 
    \end{equation*}  
    Ker po indukcijski predpostavki vemo, da ne more priti do krajšanja v produktu $a_{n -1}^{-1} b_{n -1}^{-1}$, ne more priti do krajšanja v produktu $a_{n} a_{n}$ ali njegovem inverzu $a_{n}^{-1} a_{n}^{-1}$. Enako sklepamo za preostale produkte.
    \begin{itemize}
        \item Produkt $b_{n} b_{n}$ in njegov inverz sta okrajšana, ker je okrajšan $b_{n - 1}^{-1} a_{n-1}.$
        \item Produkt $a_{n}^{-1} b_{n}$ in njegov inverz sta okrajšana, ker je okrajšan $b_{n - 1} a_{n-1}.$
        \item Produkt $b_{n} b_{n}^{-1}$ in njegov inverz sta okrajšana, ker je okrajšan $a_{n - 1}^{-1} b_{n-1}.$
        \item Produkt $a_{n}^{-1} b_{n}^{-1}$ in njegov inverz sta okrajšana, ker je okrajšan $b_{n - 1} b_{n-1}.$
    \end{itemize}     
\end{dokaz}

\begin{opomba}
Produkti oblike $a_n b_n$ oziroma njihovi inverzi $b_n ^{-1} a_n ^{-1}$ niso nujno okrajšane besede, na primer že za $n = 1$ dobimo $a_1 b_1 = b^{-1} a b a a^{-1} b a^{-1} b^{-1}$. To dejstvo bomo izkoristili v nadaljevanju pri dokazu ocene iz trditve \ref{lem_vrednost_cn}. 
\end{opomba}

Najprej se prepričajmo, da so besede $a_n$ oziroma $b_n$ elementi izpeljane grupe $F_2^{(n)}$. Najdemo jo kot razmislek na koncu dokaza leme 9 v \cite[str.~14]{Schneider_2016}.

\begin{lema}
\label{lem_besede_ab_so_elementi_izpeljane_grupe}
Za vsako nenegativno celo število $n$ so besede $a_n$ oziroma $b_n$ elementi izpeljane grupe $F_2^{(n)} \subseteq F_2 = \langle a , b \rangle$.
\end{lema}
\begin{dokaz}
    Dokaz poteka z indukcijo po $n$. Za $n = 0$ je $a_0 = a \in F_2 = F_2^{(0)}$ in $b_0 = b \in F_2 = F_2^{(0)}$. Za $n > 0$ velja $a_{n + 1} = [b_n^{-1}, a_n] \in \left[ F_2^{(n)}, F_2^{(n)} \right] = F_2^{(n + 1)}$ in $b_{n+1} = [a_{n}, b_{n}] \in  \left[ F_2^{(n)}, F_2^{(n)} \right] = F_2^{(n + 1)}$. 
\end{dokaz}

Nato ocenimo dolžino členov zaporedij $(a_{n})_n$ in $(b_{n})_n$. Pri tem je za razliko od praktično vseh prejšnjih ocen pomembnejša spodnja meja, ki nam sporoča, da so elemeti $a_n$ in $b_n$ netrivialni predstavniki
izpeljane podgrupe $F_2^{(n)}$. Posledično so te besede zakoni za vse rešljive grupe razreda rešljivosti $n$ ali manj.

\begin{lema}
\label{lem_ocena_dolzine_clenov_zaporedij_ab}
Za vsak $n \in \mathbb{N} \cup \left\{ 0\right\}$ velja $4^{n} \ge l(a_{n}) = l(b_{n}) \ge 2^{n}$.
\end{lema}
\begin{dokaz}
    Dokaz poteka z indukcijo po $n$. Za $n = 0$ očitno velja $l(a_0) = l(a) = l(b) = l(b_0) = 1$. Za $n \ge 1$ z upoštevanjem definicij zaporedij razpišemo \begin{align*}
    l(b_{n+1}) &= l(a_{n} b_{n} a_{n}^{-1} b_{n}^{-1}) \\
     &= l(a_{n} b_{n}) + l(a_{n}) + l(b_{n}) \\
     &= l(b_{n}^{-1} a_{n} b_{n} a_{n}^{-1}) \\
     &= l(a_{n + 1})
\end{align*}
Za sklep v drugi in tretji vrstici je bila potrebna lema \ref{lem_ni_krajsanj_produkti_ab} ter preprost sklep, da za besedi $w_1, w_2 \in F_2$, katerih produkt $w_1 w_2$ je okrajšana beseda, velja $l(w_1 w_2) = l(w_1) + l(w_2)$.

Iz druge vrstice sledi \begin{equation*}
    l(b_{n+1}) = l(a_n b_n) + l(a_n)  + l(b_n) \ge l(a_n)  + l(b_n) = 2l(b_n).
\end{equation*}
od koder z indukcijo dobimo $l(a_{n}) = l(b_{n}) \ge 2^{n}$. Iz tretje vrstice direktno sledi $l(b_{n+1}) \le 4 l(b_n)$
od koder z indukcijo dobimo $l(b_n) \le 4^{n}$. Mimogrede -- s tem razmislekom smo na novi način dokazali oceno \ref{trd_ocitna_meja_za_kratke_zakone_resljive_grupe}.    
\end{dokaz}

Zaradi leme \ref{lem_ocena_dolzine_clenov_zaporedij_ab} lahko dobro definiramo zaporedje naravnih števil $c_n := l(a_n) = l(b_n)$, ki nam podaja dolžine netrivialnih besed v grupi $F_2^{(n)}$.
 Pred izračunom splošnega člena uvedimo naslednjo notacijo. Funkcija $f : \mathbb{N} \cup \{ 0 \} \to \mathbb{R}$ pripada \emph{razredu $o(1)$}, če je $\lim_{n \to \infty} f(n) / n  = 0$. To so torej funkcije, ki asimptotsko gledano
 zelo malo vplivajo na oceno. 
\begin{lema}
\label{lem_vrednost_cn}
Zaporedje $c_n$ za vsako število $n \ge 0$ ustreza rekurzivni zvezi $c_{n+2} = 3 c_{n+1} + 2c_{n}$ z začetnima členoma $c_0 = 1$ in $c_1 = 4$. Od tod lahko natančno izračunamo splošni člen tega zaporedja in ocenimo zgornjo mejo z zvezama \begin{equation*}
c_{n} = \left( \frac{1}{2} + \frac{5}{2 \sqrt{17}} \right) \left( \frac{3 + \sqrt{17} }{2} \right)^{n} +  \left( \frac{1}{2} - \frac{5}{2 \sqrt{17}} \right) \left( \frac{3 - \sqrt{17} }{2} \right)^{n} \le C_1 \iota^{n} + o(1),
\end{equation*}
kjer je $\iota := (3 + \sqrt{17} ) / 2 = 3{,}5615528 \ldots $ in $C_1 = 1/ 2 + 5 / (2 \sqrt{17} ) = 1{,}1063391 \ldots$. 
\end{lema}  
\begin{dokaz}
    Dokaz je v podobni obliki podan v \cite[str.~15]{Schneider_2016} in se stalno sklicuje lemo \ref{lem_ni_krajsanj_produkti_ab}, ki preprečuje nezaželena krajšanja med stičišči besed. Po definicji zaporedij $a_n$ in $b_n$ hitro vidimo, da je $c_0 = l(a) = 1$ ter $c_1 = l([b^{-1}, a]) = 4$. Nato izrazimo
    \begin{align*}
        c_{n+2} &= l(b_{n+2}) \\
         &= l([a_{n + 1}, b_{n+1}])\\
         &= l([[b_n ^{-1}, a_{n}], [a_{n}, b_{n}]]) \\
         &= l(b_{n} ^{-1} a_{n} b_{n} \underbrace{a_{n} ^{-1} a_{n}}_{\text{se pokrajša}}  b_{n} a_{n} ^{-1} b_{n} ^{-1}) + l([a_{n}, b_{n} ^{-1}]) + l([b_{n}, a_{n}]) \\
         &= \underbrace{l(b_{n} ^{-1} a_{n} b_{n})}_{l(a_{n + 1}) - l(a_{n}^{-1}) = c_{n+1} - c_{n}}  + \underbrace{l(b_{n}) + l(a_{n} ^{-1}) + l(b_{n} ^{-1})}_{3 c_{n}}  + \underbrace{l([a_{n} , b_{n}^{-1}])}_{l(a_{n + 1}) = c_{n+1}}  + \underbrace{l([b_{n}, a_{n}])}_{l(b_{n+1}) = c_{n+1}}. 
    \end{align*}
    To nam za $n \ge 0 \cup  \left\{ 0 \right\}$ podaja želeno zvezo $c_{n+2} = 3 c_{n+1} + 2c_{n}$ skupaj z začetnima vrednostima $c_0 = 1$ in $c_1 = 4$. Temu rekurzivno podanemu zaporedju pripada matrična enačba, ki jo poenostavimo z diagonalizacijo kvadratne matrike, spodaj označene z $A$ \begin{equation*}
    \begin{bmatrix}
        c_{n+ 1}\\
        c_{n} 
    \end{bmatrix} = {\underbrace{\begin{bmatrix}
        3 & 2\\
        1 & 0
    \end{bmatrix}}_{A}}^{n}  \begin{bmatrix}
        4 \\
        1 
    \end{bmatrix} = \begin{bmatrix}
        \frac{3 - \sqrt{17} }{2} & \frac{3 + \sqrt{17} }{2}\\
        1 & 1
    \end{bmatrix} \begin{bmatrix}
        \frac{3 - \sqrt{17} }{2} & 0\\
        0 & \frac{3 + \sqrt{17} }{2}
    \end{bmatrix}^{n} 
    \begin{bmatrix}
        - \frac{1}{\sqrt{17} } & \frac{1}{2} + \frac{3}{2 \sqrt{17} }\\
        \frac{1}{\sqrt{17} } & \frac{1}{2} - \frac{3}{2 \sqrt{17} }
    \end{bmatrix}
    \begin{bmatrix}
        4 \\
        1 
    \end{bmatrix}.
    \end{equation*}  
     Druga vrstica te matrične enačbe nam podaja želeni izraz splošnega člena $c_n$. Neenakost $c_n \le C_1 \iota^n + o(1)$ je posledica dejstva, 
     da je po absolutni vrednosti največja lastna vrednost matrike $A$ enaka $\iota = (3 + \sqrt{17})  / 2 = 3{,}5615528 \ldots$. Člen, ki pripada manjši lastni vrednosti konvergira proti $0$, zato je razreda $o(1)$.
\end{dokaz} 

Direktna posledica te leme je naslednja ugotovitev za rešljive grupe.

\begin{trditev}
\label{trd_osnovna_ocena_resljive_grupe} 
 Obstaja netrivialna beseda $w \in F_2$, ki je zakon v vseh grupah razreda rešljivosti $n$ ali manj, dolžine \begin{equation*}
 l(w) \le C_1 \iota^{n} + o(1),
 \end{equation*}  
 kjer sta konstanti $C_1$ in $\iota$ enaki kot v lemi \ref{lem_vrednost_cn}.  
\end{trditev}
\begin{dokaz}
    Naj bo $G$ rešljiva grupa razreda $d$. Po prejšnji lemi obstaja netrivialna beseda $w \in F_2^{(n)}$ dolžine $C_1 \iota^{n} +o(1)$. 
    Ker je grupa $G$ rešljiva razreda $d \le n$, je $G^{(n)} = G^{(d)} = \left\{  1_G \right\}$. To pomeni, da za vsak par elementov $g, h \in G$ velja
    $w(g, h) = 1_G$, torej je $w$ zakon v grupi $G$. 
\end{dokaz}

Ta ugotovitev je asimptotsko gledano veliko boljši rezultat od trditve \ref{trd_ocitna_meja_za_kratke_zakone_resljive_grupe}.
Zdaj moramo pridobljeno znanje le še prevesti na nilpotentne grupe. Brez dokaza (najdemo ga v \cite[str.~17--18]{Schneider_2016}) bomo privzeli naslednje razmeroma znano dejstvo o členih
spodnje centralne vrste. 

\begin{lema}
\label{lem_povezava_med_spodnjo_in_izpeljano_vrsto}
Za vsako število $n \ge 0$ velja inkluzija
\begin{equation*}
G^{(n)} \subseteq \gamma_{2^{n}}(G).
\end{equation*}    
\end{lema}

Naslednja trditev je kombinacija posledice 4 in leme 11 iz naloge \cite[str.~16--17]{Schneider_2016}.

\begin{trditev}
\label{trd_koncna_ugotovitev_nilpotentne_v_nalogi}
 Obstaja netrivialna beseda $w \in F_2$, ki je zakon v vseh nilpotentnih grupah $G$ moči največ $n$, dolžine \begin{equation*}
 l(w) \le  C_3 \log(n)^{\kappa} + o(1),
 \end{equation*}  
 kjer sta $C_3 = 7{,}712869694 \ldots$ in $\kappa = \log_2(\iota) = 1{,}832506 \ldots$ konstanti.    
\end{trditev}

\begin{dokaz}
    Za vsako število $k \ge 1$ je
    število $e = \left\lceil \log_2(k) \right\rceil$ najmanjše naravno število, da velja $k \le 2^{e} < 2k$.
    Od tod po lemi \ref{lem_povezava_med_spodnjo_in_izpeljano_vrsto} sledi \begin{equation*}
    F_2^{(e)} \subseteq \gamma_{2^{e}}(F_2) \subseteq \gamma_k(F_2).
    \end{equation*}  
     Po trditvi \ref{trd_osnovna_ocena_resljive_grupe} obstaja netrivialna beseda $w \in  F_2^{(e)}$ dolžine največ $C_1 \iota^{e} + o(1)$.
    Zaradi izbira števila $e$ lahko zapišemo \begin{equation*}
    l(w) \le  C_1 \iota^{e} = C_1 2^{\log_2(\iota) e} < C_1 (2k)^{\log_2(\iota)} = C_1 \iota k^{\log_2(\iota)} = C_2  k^\kappa.
    \end{equation*}
    Naj bo grupa $G$ nilpotentna razreda $d$, moči $n$ ali manj. Zaradi nilpotentnosti $G$ za vsako celo število $i \ge 0$ iz netrivialnosti grupe $\gamma_i(G)$ sledi netrivialnost kvocienta $\gamma_i(G) / \gamma_{i + 1}(G)$, saj je centralna vrsta nilpotentnih grup pred iztekom strogo padajoča.
    Za razred nilpotentnosti $d$ velja po posledici \ref{psl_ocena_razreda_nilpotentnosti} ocena $d \le \left\lfloor \log_2(G)  \right\rfloor \le \log_2(n)$. Zato po prvem sklepu dokaza obstaja netrivialna beseda $w \in \gamma_d(G)$ dolžine \begin{equation*}
    l(w) \le C_2 d^{\kappa} \le C_2 \log_2(n)^{\kappa} = \frac{C_2}{\log_2(n)^{\kappa}} \log(n)^{\kappa} = C_3 \log(n)^{\kappa}, 
    \end{equation*}  
     saj velja $w \in \gamma_{\left\lfloor \log_2(n) \right\rfloor}(F_2) \subseteq \gamma_{d}(F_2)$. Od tod z analognim razmislekom kot v trditvi \ref{trd_osnovna_ocena_resljive_grupe} sledi, da je $w$ netrivialni zakon v vseh nilpotentnih grupah razreda $d$ ali manj. 
\end{dokaz}

V nadaljevanju članka \cite[str.~5--9]{Elkasapy_Thom_2013} avtorja eksponent $\kappa$ iz prejšnje trditve izboljšata na $\lambda := 1{,}44115577 \ldots$, pri čemer je treba namesto konstante $C_3$ vzeti faktor oblike $(8{,}395184144 \ldots + o(1))$. To storita s preučevanjem funkcije \begin{equation*}
\gamma(w) := \max \left\{ n \in  \mathbb{N}  \middle|\, w \in \gamma_{n}(F_2) \right\} \cup \left\{ \infty\right\}. 
\end{equation*}  
Če namreč definiramo $\gamma_n := \gamma(a_{n}) = \gamma(b_{n})$, lahko za vsako celo število $n \ge 0$ dokažemo zvezo $\gamma_{n + 2} - 2 \gamma_{n+1}  - \gamma_{n} \ge  0$, s čimer po enakem postopku kot v dokazu \ref{lem_vrednost_cn} ocenimo spodnjo mejo $\gamma_n \ge C_4 (1 + \sqrt{2})^{n} - o(1)$.
Avtorja razmislek zaključita z ugotovitvijo, da je namesto eksponenta $\kappa = \log_2( \iota)$ ustrezen $\lambda := \log_{1 + \sqrt{2}}(\iota)$.

Da dobimo primerljiv rezultat za rešljive grupe, se moramo precej bolj potruditi. Postopek je opisan v \cite[str.~3--4]{Thom_2015} oziroma podrobneje v \cite[str.~19--25]{Schneider_2016}. Sklicuje se na lastnosti grup avtomorfizmov nilpotentnih grup, ki jih vložimo
v primerne splošne linearne grupe. Slednjim znamo natančno omejiti razrede rešljivosti, saj so predmet klasične obravnave v teoriji grup. Ker je jedro te vložitve nilpotentno, se lahko zaradi razširitvene leme \ref{lem_razsiritvena_lema} skličemo na rezultat o dolžinah zakonov v nilpotentnih grupah \ref{trd_koncna_ugotovitev_nilpotentne_v_nalogi}, kar nam zagotovi naslednji izrek, ki ga najdemo v obliki trditve 4 v \cite[str.~25]{Schneider_2016}.  

\begin{izrek}
\label{izr_glavni_izrek_resljive}
 Za vsako število $n \in  \mathbb{N} \cup  \left\{ 0\right\}$ obstaja netrivialna beseda $w \in F_2$ dolžine \begin{equation*}
 l(w) \le (C_{10} + o(1)) \log(n)^{\lambda},
 \end{equation*}  
   ki je zakon v vseh rešljivih grupah moči $n$ ali manj, kjer sta konstanti enaki $C_{10} := 86.321{,}05422 \ldots$ in $\lambda := 4{,}331612776 \ldots$ 
\end{izrek}