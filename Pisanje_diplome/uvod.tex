\section{Uvod}

Abstraktni produkt elementov $a_1, \ldots , a_k$ ter njihovih inverzov $a_1^{-1}, \ldots , a_k^{-1}$, je \emph{$k$-črkovni zakon v grupi $G$}, če ima lastnost, da za vsako zamenjavo $a_1, \ldots, a_k$ s konkretnimi
elementi $g_1, \ldots, g_k \in G$ dobimo rezultat $1_G\in G$. Zakonu $1$ pravimo \emph{trivialni zakon}, ki v kontekstu raziskovanja zakonov ni posebej zanimiv.

Najosnovnejši primer netrivialnega dvočrkovnega zakona se pojavi pri Abelovih grupah. Grupa $G$ je namreč Abelova natanko tedaj, ko za vsaka elementa $g, h \in  G$ velja $gh = hg$, kar je ekvivalentno
zahtevi \begin{equation*}
ghg^{-1}h^{-1} = [g,h] = 1_G.
\end{equation*}
Grupa $G$ je torej Abelova natanko tedaj, ko je štiričrkovna beseda $aba^{-1}b^{-1}$ v njej zakon. 

Nadvse pomembno je vprašanje, ali vsaka grupa premore netrivialni zakon. Odgovor nanj je v splošnem negativen, kar bomo videli v nadaljevanju kot posledico trditve \ref{trd_dolzina_zakonov_za_psl2p}. 
Očitna posledica Lagrangeevega izreka pa je, da vsaka končna grupa $G$ premore
netrivialni zakon $a^{\lvert G \rvert }$, saj za vsak element $g \in G$ velja  
\begin{equation*}
g^{\lvert G \rvert } = 1_G.
\end{equation*}  

To dejstvo si natančneje oglejmo na primeru simetrične grupe $S_n$. Zanjo po Lagrangeevem izreku velja enočrkovni zakon $a^{n !}$, katerega dolžina znaša $n!$, kar je po Stirlingovi formuli približno
\begin{equation*}
n! \sim \sqrt{2 \pi n} \left( \frac{n}{e} \right)^{n}.
\end{equation*}  
Asimptotsko gledano je to zelo dolg zakon, veliko krajši je na primer že zakon oblike $a^{\exp(S_n)}$, kjer smo označili $\exp(S_n) = \text{lcm}(1, \ldots, n)$, za katerega s pomočjo osnovnega izreka o praštevilih \ref{lem_gostota_prastevil} dobimo asimptotsko oceno \begin{equation*}
\text{lcm}(1, \ldots ,n) \sim e^{n}.
\end{equation*}  
Trenutno najboljša ocena za dolžine kratkih zakonov izhaja iz članka \cite{Kozma_Thom_2016}, ki bo predstavljena v razdelku \ref{sec_simetricne_grupe}.

Na tej točki se naravno pojavi nekaj vprašanj: kako dolgi so najkrajši netrivialni zakoni za določeno grupo oziroma družino grup? Ali lahko ocenimo
asimtotsko rast dolžine najkrajših netrivialnih zakonov za družine grup, recimo za družino $\text{Sym}(n)$? Kaj pa za vse grupe moči $n$ ali manj? 
Katere družine grup se še posebej naravno pojavljajo pri takšnem raziskovanju?
Prav ta vprašanja bodo bistvo diplomske naloge, v kateri bom predstavil dosedanje
rezultate ter različne pristope, ki so jih ubrali raziskovalci. Na koncu bom predstavil,
kako lahko z uporabo računalnika dobimo vpogled v delež zakonov med besedami.

Zgodovinsko gledano so zgornja vprašanja razmeroma sodobna. Obravnavanje lastnosti zakonov namreč v nekem smislu sega že do Abela in Galoisa, 
saj lahko tako Abelove kot rešljive grupe zelo naravno karakteriziramo s pomočjo zakonov. Zakoni so pomembni tudi za obravnavo klasičnih Bursidovih problemov,
ki matematikom burijo domišljijo že od začetka 20.~stoletja. Ti problemi sprašujejo po končnosti specifičnih kvocientov prostih grup, kar bo nekoliko podrobneje razloženo v
razdelku \ref{sec_racunalnisko_iskanje}. 

% Zakone lahko recimo uporabljamo za prezentacijo grup, morda najbolj znani primer predstavljajo diedrske grupe \begin{equation*}
%     D_{2n} = \langle r, Z  \vert  r^{n} = Z^{2} = 1, (rZ)^2 = 1 \rangle.  
%     \end{equation*}  
%     Diedrska grupa je enolično določena s tremi enočrkovnimi zakoni, ki jim morata ustrezati generatorja $r$ in $Z$ ($a^{n}$ in $a^{2}$) oziroma njun produkt $rZ$ ($a^{2}$).
