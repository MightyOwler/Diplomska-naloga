\section{Iskanje zakonov z računalnikom}
\label{sec_racunalnisko_iskanje}

Na roke preverjati, ali je beseda zakon, je zoprno. Veliko lažje je z uporabo raznih lastnosti grupe -- še posebej redov -- preveriti, da beseda ni zakon. 
Zato je zelo naravno pomisliti na uporabo računalnika. % TODO tole bolje poveži 
Programi se nahajajo v repozitoriju (TODO daj pravo povezavo, lepo uredi).

\subsection{Iskanje zakonov v grupah $\operatorname{PSL}_2(p)$} %

Najprimitivnejši način iskanja zakonov v dani grupi $G$ je kar po definiciji: Poiščemo vse besede grupe $F_2 = \langle a,b \rangle$ določene dolžine, nato pa jih izvrednotimo na vseh možnih parih.
Za začetek me je zanimalo, koliko zakonov dolžine 17 ali manj premorejo grupe $\operatorname{PSL}_2(p)$. Za višje dolžine je bilo potrebno generirati nepraktično veliko besed. Pri tem se zavedamo, da grupe $\operatorname{PSL}_2(p)$ ne morejo imeti zakonov krajših od $p$-črk po trditvi \ref{trd_dolzina_zakonov_za_psl2p}.
Program sem spisal v jeziku C++, ki je v splošnem veliko hitrejši od GAP-a, opisuje ga spodnja psevdokoda.

\begin{verbatim}
    # generiranje besed in parov elementov

    za k = 1, ... , 17:
        - generiaj vse okrajšane besede dolžine k
        - shrani jih v datoteko
    
    za p = 2, 3, 5, 7, 11, 13, 17:
        - predstavi elemente grupe PSL_2(p) kot 2x2 matrike
        - generiraj vse pare elementov
        - pare shrani v datoteko

    # preverjanje zakonov
    
    za p = 2, 3, 5, 7, 11, 13, 17:
        za k = p, ... , 17:
            - preberi pare grupe PSL_2(p) in besede dolžine k 
            iz generiranih datotek
            - na vsaki besedi evalviraj vse pare
            - če je rezultat vseh evalvacij besede identična matrika,
            je ta beseda zakon
\end{verbatim}

Hitro se je izkazalo, da je tak pristop zelo neučinkovit. Problem je namreč v tem, da število besed dolžine $k$ narašča eksponentno. Če brez škode za splošnost fiksiramo prvo črko, je to število enako $3^{k - 1}$,
saj lahko v vsakem koraku dodamo natanko 3 črke, da ne pride do krajšanja. Tudi če bi obravnavali tako imenovane \emph{komutatorske besede}, ki vsebujejo enako število črk kot njihovih inverzov (število črk $a$ je enako številu črk $a^{-1}$, podobno za $b$),
število dvočrkovnih besed, ki bi jih morali pregledati, mnogo prehitro narašča, da bi lahko pokazali karkoli smiselnega.

V splošnem se sicer da oceniti, z najmanj kolikšno verjetnostjo je naključna beseda $w \in F_2$ zakon v grupi $G$.
Oceniti moramo indeks grupe zakonov $K(G, 2)$ v grupi $F_2$. S pomočjo razmisleka pod dokazom leme \ref{lem_koncni_indeks_koncnega_preseka} v primeru $k  = 2$ dobimo oceno
\begin{equation*}
    \left[ F_2 : K(G, 2) \right] \le {\lvert G \rvert}^{{\lvert G \rvert}^2},
    \end{equation*}  
kar vsekakor ni ravno spodbudno. Pa vendar se v praksi izkaže, da ti indeksi dejansko so razmeroma visoki. Članek \cite{Cocke_2020} nam ponuja konkretne vrednosti naslednjih indeksov.
\begin{align*}
    \left[ F_2 : K(D_{10}, 2) \right] &= 2^2 \cdot 5^{5} = 12500, \\
    \left[ F_2 : K(S_3, 2) \right] &= 2^2 \cdot 3^{5} = 972, \\
    \left[ F_2 : K(A_4, 2) \right] &= 2^{10} \cdot  3^{2} = 9216, \\
    \left[ F_2 : K(A_5, 2) \right] &= 2^{48} \cdot  3^{24} \cdot 5^{24} \approx 4.73 \cdot 10^{42}.   % 4738381338321616896000000000000000000000000.
\end{align*}

V splošnem ni veliko grup, za katere bi poznali točne vrednosti teh kvocientov \cite[str.~1]{Cocke_2020}. Rezultatov za grupe $\operatorname{PSL}_2(q)$ nisem našel.

\subsection{Iskanje generatorjev zakonov za nilpotentne grupe}

Kot smo videli v prejšnjem poglavju, moramo do problema pristopiti bolj zvito. Delež zakonov med vsemi dvočrkovnimi besedami nam določa kvocient \begin{equation*}
\bigslant{F_2}{\bigcap_{\varphi \in \operatorname{Hom}(F_2, G)}}.  
\end{equation*}  
Vemo že, da je grupa $F_2^{\exp(G)} = \left\{ w^{\exp(G)}  \middle|\, w \in F_2 \right\}$ edinka v $F_2$. (TODO tega v resnici nisem nikjer pokazal, je treba).
Zato lahko po tretjem izreku o izomorfizmu zapišemo 
\begin{equation*}
    \bigslant{F_2}{\bigcap_{\varphi \in \operatorname{Hom}(F_2, G)}} \cong \dfrac{\bigslant{F_2}{F_2^{\exp(G)}}}{\bigslant{ \left( \bigcap\limits_{\varphi \in \operatorname{Hom}({F_2} / {F_2^{\exp(G)}}, G)} \ker \varphi \right) }{F_2^{\exp(G)}}}.
\end{equation*}  
  
S tem smo problem poenostavili, saj nam za izračun zakonov ni več treba računati jeder vseh homomorfizmov $F_2 \to G$, temveč le še $F_2 / F_2^{\exp(G)} \to G$.
Če je grupa $G$ nilpotentna razreda $d$, uporabimo podoben razmislek. 
 (TODO to bo treba dokazati pri nilpotentnih grupah) 
Ker je poljubni člen spodnje centralne vrste edinka v $F_2$, je tudi grupa $\gamma_{d+ 1}(F_2)$.
Produkt edink je edinka, zato je tudi $F_2^{\exp(G)} \gamma_{d+ 1}(F_2)$ edinka v $F_2$, in lahko tvorimo kvocient 
\begin{equation*}
    \bigslant{F_2}{\bigcap_{\varphi \in \operatorname{Hom}(F_2, G)}} \cong \dfrac{\bigslant{F_2}{F_2^{\exp(G)}\gamma_{d+1}(F_2)}}{ \bigslant{ \left( \bigcap\limits_{\varphi \in \operatorname{Hom}({F_2} / {F_2^{\exp(G)}\gamma_{d+1}(F_2)}, G)} \ker \varphi \right) }{F_2^{\exp(G)}\gamma_{d+1}(F_2)}}.
\end{equation*}

S tem pa smo problem (za nilpotentne grupe) že močno poenostavili, saj vsebuje GAP paket za delo z nilpotentnimi grupami \texttt{nq}, s pomočjo katerega lahko zgornji kvocient izračunamo in obravnavamo kot grupo.
Na tak način sem izračunal indekse za vse nilpotentne grupe do vključno moči 64. Program je dostopen na repozitoriju (TODO tukaj prilepi ), opisuje ga naslednja psevdokoda. 

\begin{verbatim}
    vse nilpotentne grupe želenih moči shranimo v seznam 
    za vsako grupo G iz seznama:
        izračunamo vrednosti exp(G) in d
        kvocient := (zgornji izraz z ustreznimi vrednostmi)
        zakoni := presek homomorfizmov kvocient -> G
        poračunamo strukturo in velikost kvocienta kvocient/zakoni
    izračunane rezultate shranimo v datoteko
\end{verbatim}

Ta pristop do problema je mnogo boljši, saj je ne le bolj povezan s strukturo grup, temveč tudi omogoča boljši vpogled v razumevanje zakonov.
Z njegovo pomočjo je namreč lažje opaziti in posledično dokazati naslednje lastnosti zakonov. Začnimo s preprostimi.

\begin{trditev}
\label{trd_lastnosti_zakonov_ciklicne}
 Za vsako ciklično grupo $C_n$ je \begin{equation*}
 F_2 / K(C_n, 2) \cong C_n \times C_n
 \end{equation*}  
 in posledično sledi \begin{equation*}
\left[ F_2 : K(C_n, 2) \right] = n^2.
 \end{equation*}  
Z drugimi besedami, delež zakonov v cikličnih grupah med vsemi besedami je $1 / n^2$.
\end{trditev}
\begin{dokaz}
Naj bo $F_2 = \langle a, b \rangle$.
Najti moramo epimorfizem $F_2 \to C_n \times C_n$ z jedrom $K(C_n ,2)$. Na tej točki se spomnimo preprostega sklepa, da velja $K(C_n, 2) = K(C_n \times C_n, 2)$. Naj bo $\xi \in C_n$ generator ciklične grupe. Definirajmo preslikavo $\varphi: F_2 \to C_n \times C_n$,
inducirano s slikama elementov $a \mapsto (\xi, 1_{C_n})$ in $b \mapsto (1_{C_n}, \xi)$.
Ta preslikava je očitno surjektivna, preveriti moramo še, da je $\ker \varphi = K(C_n, 2)$. Najprej preverimo inkluzijo $\ker \varphi \subseteq K(C_n, 2)$.
Naj bo $w  \in \ker \varphi \subseteq  F_2$ okrajšana beseda oblike $w = a^{r_1} b^{s_1} \ldots a^{r_{k}} b^{s_k}$ za neka cela števila $r_1, s_1, \ldots , r_k , s_k$. To pomeni, da je \begin{equation*}
\varphi(w) = \varphi(a^{r_1}) \varphi(b^{s_1}) \ldots \varphi(a^{r_k}) \varphi(b^{s_k}) = \left( \xi^{r_1 + \ldots + r_k}, \xi^{s_1 + \ldots + s_k} \right) = \left( 1_{C_n} , 1_{C_n} \right).
\end{equation*}  
Z drugimi besedami, vsoti $r_1 + \ldots + r_k$ in $s_1 + \ldots + s_k$ morata biti deljivi z $n$. Zato imamo za poljubna elementa $g, h \in C_n$ \begin{equation*}
w(g, h) =  g^{r_1 + \ldots + r_k} h^{s_1 + \ldots + s_k}) = 1_{C_n},
\end{equation*}
torej je $w$ zakon v $C_n$. Dokažimo še $\ker \varphi \supseteq K(C_n, 2)$. Naj bo $w \in K(C_n, 2)$ okrajšana beseda oblike $w = a^{r_1} b^{s_1} \ldots a^{r_{k}} b^{s_k}$ za neka cela števila $r_1, s_1, \ldots , r_k , s_k$.
Potem velja \begin{equation*}
    \varphi(w) = \varphi(a)^{r_1} \varphi(b)^{s_1} \ldots \varphi(a)^{r_k} \varphi(b)^{s_k} = w(\varphi(a), \varphi(b)) = (1_{C_n} , 1_{C_n}).
    \end{equation*}
    Zadnja enakost sledi iz dejstva, da je $w$ zakon v grupi $C_n$ in posledično v $C_n \times C_n$.
\end{dokaz}
\begin{posledica}
\label{psl_lastnosti_zakonov_elementarno_abelove}
Naj bo grupa $G$ elementarno Abelova, torej $G = \prod_{i = 1}^{n} C_{p^{k_{i}}}$. Potem velja $F_2 / K(G, 2) \cong C_{p^{k}} \times C_{p^{k}}$, kjer je $k = \max_{i = 1 , \ldots , n} k_i$.
\end{posledica}
\begin{dokaz}
 Beseda $w \in F_2$ je zakon v $C_{p^{k}}$ natanko tedaj, ko je zakon v vsakem faktorju produkta $\prod_{i = 1}^{n} C_{p^{k_{i}}}$. Implikacija v levo je zato očitna, implikacija v desno pa tudi,
saj za vsak $i = 1 , \ldots , n$ velja $C_{p^{k_{i}}} \le C_{p^{k}}$, zakoni pa se prenašajo na podgrupe.
\end{dokaz}

\begin{posledica}
\label{psl_lastnosti_zakonov_splosni_produkti_ciklicnih}
Naj bo grupa $G$ poljubni končni produkt cikličnih grup. Natančneje, naj bo $G = \prod_{i = 1}^{n} \prod_{j = 1}^{n_i} C_{p_{i}^{k_{i,j}}}^{m_j}$, kjer so za vsak $i = 1, \ldots, n$ $p_i$ paroma različna praštevila, števila $n_i, m_{i} \ge 1$, in vsak $j = 1, \ldots , n_i$ števila $k_{i, j} \ge 1$ paroma različna. % TODO to se ne sliši prav
Naj bodo $k_i = \max_{j = 1, \ldots, n_i}$. Potem velja $F_2 / K(G, 2) \cong C_{l} \times C_{l}$, kjer je $l = p_1^{k_1} \cdots  p_n^{k_n}$.
\end{posledica}
\begin{dokaz}
V luči prejšnje posledice je beseda $w \in F_2$ zakon v grupi $C_l$ natanko tedaj, ko je zakon v grupi $\prod_{i = 1}^{n} C_{p^{k_i}}$, ki je po klasifikaciji končnih Abelovih grup izomorfna $C_l$.
\end{dokaz}

% TODO skušaj sklepati še kaj več

TODO % file:///C:/Users/jasak/Downloads/IJGT_2024%20Autumn_Vol%2013_Issue%203_Pages%20307-318%20(1).pdf
NAPIŠI POVEZAVO Z AMIT'S CONJECTURE

