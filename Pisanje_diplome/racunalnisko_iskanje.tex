\section{Iskanje zakonov z računalnikom}
\label{sec_racunalnisko_iskanje}

Na roke dokazati, da je beseda zakon, je -- z izjemo posebnih primerov -- zoprno. 
Zato je zelo naravno pomisliti na uporabo računalnika. 
V tem poglavju sta opisana dva različna pristopa za iskanje  oziroma razumevanje zakonov v končnih grupah. 
Programa se nahajata v repozitoriju diplomske naloge \url{https://github.com/MightyOwler/Diplomska-naloga/} v mapi \texttt{Program\_za\_racunsko\_iskanje\_zakonov}.

\subsection{Iskanje zakonov v grupah $\operatorname{PSL}_2(p)$}\label{sec_iskanje_psl2p}

Najprimitivnejši način iskanja zakonov v dani grupi $G$ je kar po definiciji: Poiščemo vse besede grupe $F_2 = \langle a,b \rangle$ določene dolžine, nato pa jih izvrednotimo na vseh možnih parih.
Za začetek me je zanimalo, koliko zakonov dolžine 17 ali manj premorejo grupe $\operatorname{PSL}_2(p)$. Za višje dolžine je bilo potrebno generirati nepraktično veliko besed. Pri tem se zavedamo, da grupe $\operatorname{PSL}_2(p)$ ne morejo imeti zakonov dolžine $p$ ali manj po trditvi \ref{trd_dolzina_zakonov_za_psl2p}.
Program sem spisal v jeziku C++, ki je v splošnem veliko hitrejši od GAP-a, opisuje ga spodnja psevdokoda.

\begin{algorithm}[ht]
    \caption{Generiranje besed in parov elementov ter preverjanje zakonov v \( \operatorname{PSL}_2(p) \)}
    \label{alg_preverjanje_zakonov}
    \raggedright
    \textbf{Vhod:} Brez vhodnih parametrov. \\
    \textbf{Izhod:} Seznam besed, ki so zakoni v \( \operatorname{PSL}_2(p) \) za določena praštevila \( p \).
  
    \begin{algorithmic}[1]
      \comment{Generiranje besed in parov elementov}
      \For{$k$}{1}{17}
        \State Generiraj vse okrajšane besede dolžine $k$
        \State Shranite jih v datoteko
      \EndFor
      
      \ForEach{$p$}{$2, 3, 5, 7, 11, 13, 17$}
        \State Predstavi elemente grupe \( \operatorname{PSL}_2(p) \) kot $2 \times 2$ matrike
        \State Generiraj vse pare elementov
        \State Pare shrani v datoteko
      \EndFor
  
      \comment{Preverjanje zakonov}
      \ForEach{$p$}{$\{2, 3, 5, 7, 11, 13, 17\}$}
        \For{$k$}{$p$}{$17$}
          \State Preberi pare grupe \( \operatorname{PSL}_2(p) \) in besede dolžine $k$ iz generiranih datotek
          \State Na vsaki besedi evalviraj vse pare
          \If{je rezultat vseh evalvacij besede identična matrika}
            \State Ta beseda je zakon
          \EndIf
        \EndFor
      \EndFor
  
    \end{algorithmic}
  \end{algorithm}


Hitro se je izkazalo, da je tak pristop zelo neučinkovit. Problem je namreč v tem, da število besed dolžine $k$ narašča eksponentno. Če brez škode za splošnost fiksiramo prvo črko, je to število enako $3^{k - 1}$,
saj lahko v vsakem koraku dodamo natanko 3 črke, da ne pride do krajšanja. Tudi če bi obravnavali zgolj tako imenovane \emph{komutatorske besede}, ki vsebujejo enako število črk kot njihovih inverzov (število črk $a$ je enako številu črk $a^{-1}$, podobno za $b$),
njihovo število mnogo prehitro narašča, da bi lahko pokazali karkoli smiselnega.

V splošnem se sicer da oceniti, z najmanj kolikšno verjetnostjo je naključna beseda $w \in F_2$ zakon v grupi $G$.
Oceniti moramo indeks grupe zakonov $K(G, 2)$ v grupi $F_2$. S pomočjo posledice \ref{psl_koncni_indeks_preseka} v primeru $k  = 2$ dobimo oceno
\begin{equation*}
    \left[ F_2 : K(G, 2) \right] \le {\lvert G \rvert}^{{\lvert G \rvert}^2},
    \end{equation*}  
kar vsekakor ni ravno spodbudno. Pa vendar se v praksi izkaže, da ti indeksi dejansko so razmeroma visoki. Članek \cite{Cocke_2020} nam ponuja konkretne vrednosti naslednjih indeksov.
\begin{align*}
    \left[ F_2 : K(D_{10}, 2) \right] &= 2^2 \cdot 5^{5} = 12500, \\
    \left[ F_2 : K(\text{Sym}(3), 2) \right] &= 2^2 \cdot 3^{5} = 972, \\
    \left[ F_2 : K(\text{Alt}(4), 2) \right] &= 2^{10} \cdot  3^{2} = 9216, \\
    \left[ F_2 : K(\text{Alt}(5), 2) \right] &= 2^{48} \cdot  3^{24} \cdot 5^{24} \approx 4.73 \cdot 10^{42}.   % 4738381338321616896000000000000000000000000.
\end{align*}

V splošnem ni veliko grup, za katere bi poznali točne vrednosti teh kvocientov \cite[str.~1]{Cocke_2020}. Podobnih rezultatov za grupe $\operatorname{PSL}_2(q)$ nisem našel. Ker so te grupe za $q > 3$ enostavne (\cite{Jezernik_2023}), si za njihovo obravnavo
z ugotovitvami naslednjega razdelka ne moremo pomagati.

\subsection{Iskanje zakonov v nilpotentnih grupah}

Kot smo videli v prejšnjem razdelku, moramo do problema pristopiti bolj zvito. Delež zakonov med vsemi dvočrkovnimi besedami nam določa kvocient \begin{equation*}
\bigslant{F_2}{\bigcap_{\varphi \in \operatorname{Hom}(F_2, G)}} \operatorname{ker} \varphi.  
\end{equation*}  
Ni težko videti, da je grupa $F_2^{\exp(G)} =  \langle w^{\exp(G)}  \vert \, w \in F_2 \rangle$ edinka v $F_2$. Za poljubne besede $w_i \in F_2, u \in  F_2$ in celo število $n$ namreč velja $(\prod_{i} w_i^n)^u = \prod_{i} (w_i^u)^n$,
tudi v primeru $n = \exp(G)$.
Ni težko razmisliti, da je vsaka beseda $w \in  F_2^{\exp(G)}$ zakon v grupi $G$, torej $F_2^{\exp(G)} \subseteq K(G, 2)$. Za vsak par $g, h \in G$ namreč velja \begin{equation*}
w(g, h) =  \prod_{i = 1}^{n} w_i(g,h)^{\exp(G)} = \prod_{i = 1}^{n} g_i^{\exp(G)} = 1_G.
\end{equation*}  
Po tretjem izreku o izomorfizmu zapišemo
\begin{equation*}
    \bigslant{F_2}{\bigcap_{\varphi \in \operatorname{Hom}(F_2, G)}} \operatorname{ker} \varphi \cong \dfrac{\bigslant{F_2}{F_2^{\exp(G)}}}{\bigslant{ \left( \bigcap\limits_{\varphi \in \operatorname{Hom}({F_2} / {F_2^{\exp(G)}}, G)} \ker \varphi \right) }{F_2^{\exp(G)}}}.
\end{equation*}  
Tu se pojavi nezanemarljiv problem: v splošnem nam nič ne zagotavlja končnosti kvocienta $B(2, \exp(G)) := F_2 / {F_2}^{\exp(G)}$. 
Pravzaprav smo prišli do klasičnega Burnsidovega problema, ki sprašuje po končnosti kvocientov oblike $B(m, n) := F_m / F_m^n$.
Po rezultatu Lysenoka leta 1996 recimo velja, da je grupa $B(2, n)$ neskončna za $n \ge 8000$ (\cite[str.~2]{Vaughan-Lee_1999}). 
Da lahko vseeno nadaljujemo s podobnim razmislekom, se osredotočimo na nilpotentne grupe. 
Potrebovali bomo naslednjo lemo, idejo za dokaz najdemo v \cite[str.~13--14]{Segal_1983}.

\begin{lema}
\label{lem_koncnogenerirana_nilpotentna}
Naj bo $H$ končno generirana nilpotentna grupa razreda $d$ in naj ima končni eksponent. Potem je $H$ končna.
\end{lema}
\begin{dokaz}
Za začetek pokažimo, da so grupe $\gamma_{i}(H) / \gamma_{i + 1}(H)$ Abelove za vsa števila $i \ge 1$. To sledi iz preprostega sklepa \begin{equation*}
\gamma_{i + 1}(H) = [\gamma_i(H), H] \supseteq [\gamma_i(H), \gamma_i(H)].
\end{equation*}
Nato dokažimo, da je vsaka končno generirana Abelova grupa končnega eksponenta končna. Naj bo $K = \langle k_1, \ldots, k_m \rangle$ grupa, ki ustreza tem zahtevam.
Ker lahko zaradi komutativnosti vsak element $k \in  K$ zapišemo v obliki \begin{equation*}
k = k_1^{n_1} \cdots k_m^{n_m}
\end{equation*}  
za neka števila $0 \le n_i < \exp(K)$, je $K$ res končna.
Tako smo razmislili, da so končne vse grupe $\gamma_{i}(H) / \gamma_{i + 1}(H)$ za $i \ge 1$. Ker je $H$ nilpotentnta razreda $H$ in velja \begin{equation*}
    [H : \gamma_2(H)] [\gamma_2(H) : \gamma_3(H)] \cdots [\gamma_{d}(H) : \gamma_{d + 1}(H)]  < \infty,
    \end{equation*}  
    sklepamo, da je $H$ končna.
\end{dokaz}

Predpostavimo, da je grupa $G$ nilpotentna razreda $d$. Podobno kot v dokazu \ref{trd_osnovna_ocena_resljive_grupe} sklepamo, da so vse besede iz $\gamma_{d+1}(F_2)$
zakoni v grupi $G$. Ker velja $\gamma_{d+1}(G) = \left\{ 1_G \right\}$, bo za vsako besedo $w \in \gamma_{d+1}(F_2)$ in vsak par elementov $g ,h \in G$ veljalo $w(g,h) = 1_{G}$.
Tako imamo $\gamma_{d+1}(F_2) \subseteq K(G, 2)$.
Ker je poljubni člen spodnje centralne vrste edinka v $F_2$, je edinka tudi grupa $\gamma_{d+ 1}(F_2)$.
Produkt edink je edinka, zato je tudi $F_2^{\exp(G)} \gamma_{d+ 1}(F_2)$ edinka v $F_2$, in lahko tvorimo kvocient 
\begin{equation*}
    \bigslant{F_2}{\bigcap_{\varphi \in \operatorname{Hom}(F_2, G)}}  \ker \varphi \cong \dfrac{\bigslant{F_2}{F_2^{\exp(G)}\gamma_{d+1}(F_2)}}{ \bigslant{ \left( \bigcap\limits_{\varphi \in \operatorname{Hom}({F_2} / {F_2^{\exp(G)}\gamma_{d+1}(F_2)}, G)} \ker \varphi \right) }{F_2^{\exp(G)}\gamma_{d+1}(F_2)}}.
\end{equation*}

S tem smo problem v primeru nilpotentnih grup poenostavili, saj nam za izračun zakonov ni več treba računati jeder vseh homomorfizmov $F_2 \to G$, temveč le še $F_2 / F_2^{\exp(G)}\gamma_{d+1}(F_2) \to G$.
To je precej bolj ugodno, saj je grupa $F_2 / F_2^{\exp(G)}\gamma_{d+1}(F_2)$ nilpotentna razreda največ $d$ zaradi sklepa $\gamma_{d+1}(F_2) \subseteq  F_2^{\exp(G)}\gamma_{d+1}(F_2)$.
Posledično je po lemi \ref{lem_koncnogenerirana_nilpotentna} končna.
Računalniška konstrukcija tega kvocienta ni posebej zahtevna, saj GAP vsebuje paket za delo z nilpotentnimi grupami \texttt{nq} (\cite{nq2.5.11}), s pomočjo katerega ga lahko izračunamo in preučujemo njegovo grupno strukturo.
Na tak način sem izračunal indekse za vse nilpotentne grupe do vključno moči 64, za višje vrednosti je bila časovna zahtevnost prevelika. Program in rezultati so objavljeni na repozitoriju \url{https://github.com/MightyOwler/Diplomska-naloga}.
Program opisuje naslednja psevdokoda. 

% \begin{algorithm}[ht]
%     \caption{Izračun vrednosti in struktur za nilpotentne grupe}
%     \label{alg_nilpotentne_grupe}
%     \raggedright
%     \textbf{Vhod:} Seznam nilpotentnih grup želenih moči. \\
%     \textbf{Izhod:} Rezultati izračunanih struktur in velikosti kvocientov.
  
%     \begin{algorithmic}[1]
%       \State Vse nilpotentne grupe želenih moči shranimo v seznam.
%       \ForEach{$G$}{seznam}
%         \State Izračunamo vrednosti $\exp(G)$ in $d$
%         \State kvocient $\gets$ (zgornji izraz z ustreznimi vrednostmi)
%         \State zakoni $\gets$ presek jeder homomorfizmov: kvocient $\to G$
%         \State Poračunamo strukturo in velikost kvocienta $\text{kvocient}/\text{zakoni}$
%       \EndFor
%       \State Izračunane rezultate shranimo v datoteko.
%     \end{algorithmic}
%   \end{algorithm}

Ta pristop do problema je mnogo boljši od pristopa v razdelku \ref{sec_grupe_psl2q}, saj je ne le bolj povezan s strukturo grup, temveč tudi omogoča boljši vpogled v splošno razumevanje zakonov.
Z njegovo pomočjo je namreč lažje opaziti in posledično dokazati naslednje lastnosti zakonov. Začnimo s preprostimi.
\begin{trditev}
\label{trd_lastnosti_zakonov_ciklicne}
 Za vsako ciklično grupo $C_n$ je \begin{equation*}
 F_2 / K(C_n, 2) \cong C_n \times C_n
 \end{equation*}  
 in posledično sledi \begin{equation*}
\left[ F_2 : K(C_n, 2) \right] = n^2.
 \end{equation*}  
Z drugimi besedami, delež zakonov med vsemi besedami v cikličnih grupah je $1 / n^2$.
\end{trditev}
\begin{dokaz}
Naj bo $F_2 = \langle a, b \rangle$.
Najti moramo epimorfizem $F_2 \to C_n \times C_n$ z jedrom $K(C_n ,2)$. Na tej točki se spomnimo preprostega sklepa, da velja $K(C_n, 2) = K(C_n \times C_n, 2)$. Naj bo $\xi \in C_n$ generator te ciklične grupe. Definirajmo preslikavo $\varphi: F_2 \to C_n \times C_n$,
inducirano s slikama elementov $a \mapsto (\xi, 1_{C_n})$ in $b \mapsto (1_{C_n}, \xi)$.
Ta preslikava je očitno surjektivna, preveriti moramo še, da je $\ker \varphi = K(C_n, 2)$. Najprej preverimo inkluzijo $\ker \varphi \subseteq K(C_n, 2)$.
Naj bo $w  \in \ker \varphi \subseteq  F_2$ okrajšana beseda oblike $w = a^{r_1} b^{s_1} \ldots a^{r_{k}} b^{s_k}$ za neka cela števila $r_1, s_1, \ldots , r_k , s_k$. To pomeni, da je \begin{equation*}
\varphi(w) = \varphi(a^{r_1}) \varphi(b^{s_1}) \ldots \varphi(a^{r_k}) \varphi(b^{s_k}) = \left( \xi^{r_1 + \ldots + r_k}, \xi^{s_1 + \ldots + s_k} \right) = \left( 1_{C_n} , 1_{C_n} \right).
\end{equation*}  
Z drugimi besedami, vsoti $r_1 + \ldots + r_k$ in $s_1 + \ldots + s_k$ morata biti deljivi z $n$. Zato imamo za poljubna elementa $g, h \in C_n$ \begin{equation*}
w(g, h) =  g^{r_1 + \ldots + r_k} h^{s_1 + \ldots + s_k} = 1_{C_n},
\end{equation*}
torej je $w$ zakon v $C_n$. Dokažimo še $\ker \varphi \supseteq K(C_n, 2)$. Naj bo $w \in K(C_n, 2)$ okrajšana beseda oblike $w = a^{r_1} b^{s_1} \ldots a^{r_{k}} b^{s_k}$ za neka cela števila $r_1, s_1, \ldots , r_k , s_k$.
Potem velja \begin{equation*}
    \varphi(w) = \varphi(a)^{r_1} \varphi(b)^{s_1} \ldots \varphi(a)^{r_k} \varphi(b)^{s_k} = w(\varphi(a), \varphi(b)) = (1_{C_n} , 1_{C_n}).
    \end{equation*}
    Zadnja enakost sledi iz dejstva, da je $w$ zakon v grupi $C_n$ in posledično v $C_n \times C_n$.
\end{dokaz}
\begin{posledica}
\label{psl_lastnosti_zakonov_elementarno_abelove}
Naj bo grupa $G$ elementarno Abelova, torej $G = \prod_{i = 1}^{n} C_{p^{k_{i}}}$. Potem velja $F_2 / K(G, 2) \cong C_{p^{k}} \times C_{p^{k}}$, kjer je $k = \max_{i = 1 , \ldots , n} k_i$.
\end{posledica}
\begin{dokaz}
 Beseda $w \in F_2$ je zakon v $C_{p^{k}}$ natanko tedaj, ko je zakon v vsakem faktorju produkta $\prod_{i = 1}^{n} C_{p^{k_{i}}}$. Implikacija v levo je zato očitna, implikacija v desno pa tudi,
saj za vsak $i = 1 , \ldots , n$ velja $C_{p^{k_{i}}} \le C_{p^{k}}$, zakoni pa se prenašajo na podgrupe.
\end{dokaz}

\begin{posledica}
\label{psl_lastnosti_zakonov_splosni_produkti_ciklicnih}
Naj bo končna grupa $G$ Abelova. Natančneje, naj bo v skladu s klasifikacijo končnih Abelovih grup oblike $G = \prod_{i = 1}^{n} \prod_{j = 1}^{n_i} C_{p_{i}^{k_{i,j}}}^{m_j}$, kjer so za vsak $i = 1, \ldots, n$ $p_i$ paroma različna praštevila, števila $n_i, m_{i} \ge 1$, in vsak $j = 1, \ldots , n_i$ števila $k_{i, j} \ge 1$ paroma različna. % TODO to se ne sliši prav
Naj bodo $k_i = \max_{j = 1, \ldots, n_i} k_{i,j}$. Potem velja $F_2 / K(G, 2) \cong C_{e} \times C_{e}$, kjer je $e = p_1^{k_1} \cdots  p_n^{k_n}  = \exp(G)$.
Od tod sledi, da delež zakonov v Abelovih grupah znaša natanko $1 / \exp(G)^2$.
\end{posledica}
\begin{dokaz}
V luči prejšnje posledice je beseda $w \in F_2$ zakon v grupi $C_e$ natanko tedaj, ko je zakon v grupi $\prod_{i = 1}^{n} C_{p^{k_i}}$, ki je po klasifikaciji končnih Abelovih grup izomorfna $C_e$.
Nato uporabimo trditev \ref{trd_lastnosti_zakonov_ciklicne}.
\end{dokaz}
Intuitivno je to še lažje videti z naslednjim neformalnim razmislekom.
Naj bo podana beseda $w \in F_2 = \langle a, b \rangle$. Če hočemo preveriti, ali je zakon v Abelovi grupi $G$, se lahko pretvarjamo, da črke med seboj komutirajo. Tako $w$ prevedemo na besedo oblike $w' = a^r b^s$,
kjer $r$ in $s$ predstavljata vsoto eksponentov črk $a$ oziroma $b$ v besedi $w$. Beseda $a^r$ je zakon v grupi $G$ natanko tedaj, ko je $\exp(G)$ delitelj števila $r$. Zato je verjetnost, da bo $w$ zakon po črki $a$ enaka $1 / \exp(G)$. %(projekcija $Z(G, w')$ na prvo komponento mora biti enaka $G$)
Ker enako velja za $b$ in sta evalvaciji $a$ in $b$ medsebojno neodvisni, je skupna verjetnost enaka $1 / \exp(G)^2$.

% Zanimiva je tudi povezava s tako imenovano Amit–Ashurstino domnevo \cite{Camina_Cocke_Thillaisundaram_2023}. Ta trdi, da je verjetnost, da poljubna   

% TODO skušaj sklepati še kaj več

% TODO % file:///C:/Users/jasak/Downloads/IJGT_2024%20Autumn_Vol%2013_Issue%203_Pages%20307-318%20(1).pdf
% NAPIŠI POVEZAVO Z AMIT'S CONJECTURE

