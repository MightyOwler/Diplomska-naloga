\section{Zaključek}
% TODO popravi zaključek
%      G.~Amir, G.~Blachar, M.~Gerasimova in G.~Kozma, \emph{Probabilistic laws on infinite groups}, verzija 18.~4.~2023 [ogled 29.~2.~2024], dostopno na \url{https://arxiv.org/abs/2304.09144}. 
%     \end{thebibliography}

Tekom naloge smo videli, kako potekajo raznorazne konstrukcije kratkih zakonov, tako konstruktivne kot z uporabo naključnih sprehodov.
Upam, da mi je uspelo jasno pokazati, zakaj nas zanimajo prav nilpotentne, rešljive, enostavne in simetrične grupe in kako se naravno pojavijo
kot zaporedje osnovnih sklepov prek uporabe razširitvene in komutatorske leme.

Za konec predstavimo še nekaj vprašanj za nadaljnje raziskovanje. \begin{itemize}
    \item Ali se da natančno določiti grupe, v katerih obstajajo netrivialni zakoni? Tega problema se dotakne članek \cite{Schleimer_2001},
    ki pokaže, da zakoni obstajajo v vseh grupah, katerih rast glede na neko generatorsko podmnožico je polinomska. Po Gromovem izreku \cite{Gromov_1981} so to namreč
    virtualno nilpotentne grupe, ki po razširitveni lemi imajo netrivialne zakone.
    \item Ali bi lahko uporabili naključne sprehode za analizo nilpotentnih oziroma rešljivih grup (in ali je to sploh smiselno)?
    \item Na katerih družinah grup bi lahko smiselno uproabili računalniško konstrukcijo iz zadnjega poglavja?
    \item Katere druge ugotovitve o deležu zakonov med besedami lahko dokažemo s pomočjo izračunanih kvocientov?
    \item Kako bi lahko naše znanje bolj tesno povezali z Burnsidovimi problemi?
    \item Ali bi lahko naše znanje uporabili za napad Amit–Ashurstine domneve (glej \cite{Chilebus_Cocke_Ho_2024})?
\end{itemize}